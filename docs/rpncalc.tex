\documentclass{article}
\usepackage{times}
\usepackage[fancyhdr]{latex2man}

\begin{document}
\setVersion{0.1}
\begin{Name}{1}{rpncalc}{James Tappin}{RPN Scientific Calculator}{RPNcalc\\--\\A Reverse-Polish notation scientific calculator}

  \Prog{rpncalc} \URL{https://github.com/jtappin/RPNcalc} is a fully
  functional scientific calculator emulator, written entirely in
  Fortran. It is both a usable tool and a demonstration of the
  gtk-fortran libraries \URL{https://github.com/jerryd/gtk-fortran}.
\end{Name}

\section{SYNOPSIS}

  \Prog{rpncalc} \oOpt{options} \Dots

  \section{DESCRIPTION}

  A scientific decktop calculator using Reverse Polish notation.

  \section{OPTIONS}

  \begin{description}
  \item [\Opt{-c}, \Opt{-\,-closed}] start the
    calculator with the stack and memory display hidden.
  \item [\Opt{-o}, \Opt{-\,-open}] start the calculator
    with the stack and memory display visible (this is the default).
  \item [\Opt{-r}, \OptArg{-\,-restore}{ file}]
    Load the stack and memory contents from the given file.
  \item [\Opt{-m}, \OptArg{-\,-registers }{ number}]
    Set the number of memory registers to create. The default is 10.
  \item [\Opt{-R}, \Opt{-\,-radians}] Set the trigonometry unit to
    radians.
  \item [\Opt{-D}, \Opt{-\,-degrees}] Set the trigonometry unit to
    degrees.
  \item[\Opt{-G}, \Opt{-\,-grads}] Set the trigonometry unit to grads
    (1/100 of a right angle).
  \item [\Opt{-h}, \Opt{-\,-help}] Print a help text
    and exit.
  \end{description}

  \section{User's Guide}

  \subsection{Introduction}

  The gtk-fortran RPN calculator is a fully functional scientific
  calculator emulator. It is not based on any particular model of
  calculator, but rather assembles those features that I find useful in
  a calculator. It uses the stack-based reverse Polish logic system for
  2 reasons:
  \begin{enumerate}
  \item That's what I like to use, and most calculator emulators are
    algebraic notation.
  \item I'm lazy and it's a lot easier to implement a reverse Polish
    system.
  \end{enumerate}

  It is intended to be both a useful tool, and a demonstration of what
  can be done using gtk-fortran (and in particular the fortran-only
  high-level routines). The calculator is entirely written in
  Fortran~2003 (with a few 2008 features such as the
  \texttt{execute\_command\_line} subroutine.

  \subsection{Entering values:}

  All values with RPNcalc are stored as double-precision reals (Gtk+
  has no support for long doubles).

  Values can be entered either using the keypad, or by typing into the
  entry box. Values entered from the keypad make sanity checks for 2
  decimal points or a decimal entered after the exponent has been
  started and the change-sign key works in a reasonably intelligent
  way. When values are typed or pasted into the entry box, characters
  that cannot be part of a number are rejected with a warning.

  A value may be transferred from the entry to the stack by pressing
  the keyboard \texttt{Enter} key while focus is on the entry window,
  or by clicking the \textbf{Enter} key on the keypad. The \textbf{Dup}
  key copies the entry box to the stack without clearing the entry
  box. If the contents of the entry box are not a valid number (i.e. a
  Fortran \texttt{READ} statement cannot convert it to a floating point
  value) a message is displayed in the status bar and you may edit the
  entry box to correct the problem.

  \subsection{Operators:}

  The operators (\textbf{+}, \textbf{-}, \textbf{*}, \textbf{/} and
  \textbf{y**x} and also the \textbf{atan2} function) operate on the entry
  box and the top element on the stack if there is anything in the
  entry box. If the entry box is empty, then they operate on the top 2
  elements of the stack.

  Operators may also be accessed by typing the operator into the entry
  window (N.B. The exponentiation operator is \texttt{\Circum} rather than
  \texttt{**} for convenience of implementation). The addition and
  subtraction operators will only work in this way if a sign would not
  be a valid part of a number where they are entered, notably
  \texttt{+} or \texttt{-} in an empty entry box is not treated as an
  operator. Note also that operators at the end of a multi-character
  paste are not accepted.

  The result is placed on the top of the stack, and displayed in the
  result window.

  \subsection{Functions:}

  The functions operate on a single value, which is taken from the
  entry box if that has content or from the top of the stack
  otherwise. The result is placed on the top of the stack, and
  displayed in the result window.

  If the \textbf{Inverse} checkbox is set, then functions are replaced
  by their inverses (e.g. \textbf{sin} becomes \textbf{asin}), in
  addition, the \textbf{y**x} operator becomes the
  corresponding root and the
  \emph{roll down} button becomes \emph{roll up}. Note that:
  \begin{enumerate}
  \item The less-used functions in the pulldown are not affected by
    this, as none have meaningful inverses.
  \item The power operator \textbf{\Circum} entered from the keyboard is not
    converted to a root.
  \end{enumerate}

  The \textbf{Rad}, \textbf{Deg} and \textbf{Grad} radio buttons are
  used to select Radians, Degrees or Grads for the trigonometric
  functions.

  The \textbf{HMS} key is not a proper function, it doesn't remove or
  add anything to the stack. It displays the contents of the entry box
  or the top of the stack as if it were a number of hours converted to
  \emph{HH}\textbf{:}\emph{MM}\textbf{:}\emph{SS.sss} format (or
  \emph{D}\textbf{d}~\emph{HH}\textbf{h}~\emph{MM}\textbf{m}~\emph{SS.sss}\textbf{s}
  if the value is greater than 24).

  Some less-used functions are in the \textbf{More} pulldown. The
  \textbf{atan2} function computes \texttt{arctan(y/x)} removing the
  quadrant ambiguities.

  Functions whose arguments are out of range will produce an error
  message and the stack is left unchanged.

  \subsection{Stack operations}

  \begin{itemize}
  \item \textbf{CE} clears the entry box, or if that is empty deletes
    the top entry on the stack.

  \item \textbf{CA} clears the entry box and all entries on the stack.

  \item The \emph{up} button moves the selected item in the stack up
    one place. If the top item (or nothing) is selected then it is
    exchanged with the entry box.

  \item The \emph{down} button moves the selected entry on the stack
    down one place.

  \item The \emph{roll down} button, moves the last element of the
    stack to the top and all others down one place.
  \end{itemize}

  \subsection{Constants:}

  There are a number of built in fundamental physics constants that are
  build in to the calculator, these can be entered from the
  \textbf{Phys} pull-down menu.

  \subsection{Memory Registers:}

  The calculator also has memory registers (numbered from 0). The
  contents of these registers can be viewed by selecting the
  ``registers'' tab. The default number of registers is 10, but this
  may be set with the \textbf{-m} option, or by setting a new number in
  the spin box at the bottom of the registers tab.

  These can be accessed in one of two ways:
  \begin{enumerate}
  \item Select a register in the registers tab, and then click a memory
    operation. In this case the value used will be the entry box or the
    top of the stack if the entry is empty.
  \item Enter a register number in the entry box and click the memory
    operation. The value used is the top of the stack.
  \end{enumerate}

  The operations are:
  \begin{itemize}
  \item \textbf{STO:} Store the value in the selected register.
  \item \textbf{RCL:} Copy the selected register to the top of the
    stack
  \item \textbf{M+:} Add the value to the selected register
  \item \textbf{M-:} Subtract the value from the selected register
  \item \textbf{MCL:} Clear the selected register.
  \item \textbf{MCA:} Clear all registers

  \end{itemize}

  \subsection{Statistics:}

  If the \textbf{Live stats} toggle is enabled, then a summary of the
  statistical properties of the contents of the stack is maintained in
  the "Statistics" tab of the display area. 

  \subsection{Save \& Restore:}

  The stack, registers and entry box can be saved to and restored from
  a text file with the \textbf{File-$>$Save} and
  \textbf{File-$>$Restore} menu items.

  The file format is a plain text file with the floating point values
  written in hexadecimal -- this allows the retention of full-precision
  but is endian-independent. Obviously any machines that do not use
  IEEE floating point values will not be able to read files from other
  machines. Also any machine with a \texttt{c\_double} that is not
  8-bytes will not work.

  \subsection{Settings:}

  In the current version, there are 2 user-definable settings accessed
  through the \textbf{Edit} menu:

  \subsubsection{Result Format:}
  Specify the format to use in the result box. You have the options to
  select one of the standard formats:
  \begin{description}
  \item [Fixed:] A fixed number of decimal places (set in the precision
    spin box). The actual format used is "(F0.\emph{n})". WARNING: this
    may be a GNU extension.
  \item[Sci:] Scientific format. Specify the number of decimals, and
    the width of the exponent in the spin boxes. The total width is
    calculated automatically.
  \item[Eng:] Engineering format, similar to scientific, except that
    the exponent is always a multiple of 3.
  \item[Free:] Use a list-directed write (the default).
  \end{description}
  Alternatively you can type an explicit Fortran format statement into
  the combo box (with or without the enclosing parentheses). This may
  be any Fortran formatting code valid for a REAL type. Setting it to
  "*" or an empty string will use the default list-directed output (as
  will an invalid format).

  \subsubsection{Show degrees:}
  If this is enabled, then use angular rather than time notation for the
  HMS display (the button will be relabelled \textbf{DMS}).

\subsection{Cut \& Paste:}

The \textbf{Edit} menu has options to cut or copy the selected text in
the entry window (or the result window in the case of copy) to the
clipboard. The current clipboard item my also be pasted into the entry
box, or the selected text may be deleted. The usual keybindings for
these operations are also available.

\subsection{Help system:}

This manual can be accessed in a number of ways:
\begin{enumerate}
\item A manpage is generated and can be accessed using 
\texttt{man rpncalc}.
\item The ``Help'' item in the menu will display either the text
  version or (if the environment variable \texttt{RPNCALC\_VIEWER} is
  set to a PDF viewer) the PDF version.
\item The text and PDF versions are installed in the
  \texttt{share/docs/rpncalc} subdirectory of the installation directory.
\end{enumerate}

\subsection{Accelerators:}

The menu items have accerators to save mouse clicking:

\begin{itemize}
\item Save -- ctrl-s
\item Restore -- ctrl-o
\item Quit -- ctrl-q
\item Cut -- ctrl-x
\item Delete -- ctrl-shift-x
\item Copy -- ctrl-c
\item Paste -- ctrl-v
\item Set Format -- ctrl-f
\item Help -- ctrl-h
\item About -- ctrl-a
\item About gtk-fortran -- ctrl-shift-a
\end{itemize}

\section{AUTHORS}

James Tappin (jtappin at gmail dot com).

\section{LICENCE}

RPNcalc is free software and may be modified and redistributed under
the terms of the GNU General Public Licence Version 3.

\LatexManEnd
\end{document}
